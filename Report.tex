% Preview source code from paragraph 0 to 26


\title{Document Summarizer}

\author{Akshat Choube\thanks{111501031,CSE IIT PKD}, Adrian McDonald Tariang\thanks{111501001, CSE IIT PKD},
Harsh Yadav\thanks{111501009, CSE IIT PKD}}
\maketitle
\begin{abstract}
The Report describes a Document Summarizer Mini-Project as part of
Paradigms of Programming(CS 3100) for Academic Session Aug 2017 to
Nov 2017. 
\end{abstract}

\section{Introduction }

A Document Summarizer is a tool which for a given input text file
generates a output containing a list of sentences which better describes
our original file. We began with eliminating unimportant words (such
as stop words and punctuation ). We then calculate the term frequency
of every important word (\{words\}\textbackslash{}\{unimportant words\}),
and vectorize our sentences. We then cluster similar sentences together.
In the process, we ignore the sentences which have nearly zero or
near complete similarity. This is done to not let outliers affect
our clustering. We then choose the sentences that describe the cluster
best. An ordered collection of these sentences will be our summary
of the input file.(detailed explanation is given in algorithm section)

\section{Algorithm}
\begin{enumerate}
\item Read the input file. 
\item Using nltk, tokenize document into sentences.
\item For each sentence in the document :
\item Tokenize the sentence into words. 
\item Filter out unimportant words.
\item Calculate number of unique important words (Say n).
\item Create an n-dimensional vector such that if sentence contains n-th
unique word, then the n-th entry of the vector = frequency of unique
word in the file. 
\item Use k-means clustering to cluster vectors lying close in the n-dimensional
vector space to be together. 
\item For each cluster :
\item Choose the sentences which best relate with other sentences in the
same cluster. This is done in the following manner :
\item Define a score which quantifies how close two sentences are in terms
of the important words. 
\item If the score doesn\textquoteright t lie between the lower and upper
bound (ie. 20\% to 90\% similarity), ignore the score. 
\item Else, return the score.
\item Overall score of a sentence is the sum of all scores computed by comparing
this sentence with all other sentences of the cluster.
\item Sentences with higher overall score than the others are chosen as
the summary of the cluster
\item Summary of the document is the summary of all clusters in order of
occurrence in the input file.
\end{enumerate}

\section{Contribution and Timeline}

After taking up the project, we started off by discussing the algorithms
to implement summarization. Primarily, we agreed to work in parallel
on three different algorithms, and eventually take up the one that
performed the best amongst them. While Harsh Yadav took the approach
of assigning important words weights according to their length, and
choosing the highest weighted sentence. Adrian McDonald Tariang took
up Term Frequency Inverse Document Frequency (TFIDF) approach. Akshat
Choube came up with the idea of computing scores for each sentence,
based on its similarity with other sentences in the document. Each
approach gave us a better idea, which led us to the final algorithm.

After the Interim Progress Evaluation, based on suggestions by Prof.
Mrinal Das, made us implement clustering to cluster similar sentences
and choosing the ones that best describe the cluster. This ensures
that the sentences which are dissimilar to other sentences, but are
important are also considered.

The algorithm suggested by Akshat seemed to be working better than
the rest, which motivated us to carry forward with his algorithm.
However, we were yet to include clustering in the code. Harsh came
up with the idea to vectorize sentences and apply k-means clustering,
while Akshat implemented the same. The testing was performed by Adrian
who reported bugs and fixed them. Some of these include: the output
sentences weren\textquoteright t in order. Furthermore, he suggested
to include term frequency while vectorization.

Conclusively, the submitted project had efforts pitched in by all
the team members, who eventually made it possible.
